\documentclass[]{article}
\usepackage{lmodern}
\usepackage{amssymb,amsmath}
\usepackage{ifxetex,ifluatex}
\usepackage{fixltx2e} % provides \textsubscript
\ifnum 0\ifxetex 1\fi\ifluatex 1\fi=0 % if pdftex
  \usepackage[T1]{fontenc}
  \usepackage[utf8]{inputenc}
\else % if luatex or xelatex
  \ifxetex
    \usepackage{mathspec}
    \usepackage{xltxtra,xunicode}
  \else
    \usepackage{fontspec}
  \fi
  \defaultfontfeatures{Mapping=tex-text,Scale=MatchLowercase}
  \newcommand{\euro}{€}
\fi
% use upquote if available, for straight quotes in verbatim environments
\IfFileExists{upquote.sty}{\usepackage{upquote}}{}
% use microtype if available
\IfFileExists{microtype.sty}{%
\usepackage{microtype}
\UseMicrotypeSet[protrusion]{basicmath} % disable protrusion for tt fonts
}{}
\usepackage[margin=1in]{geometry}
\usepackage{color}
\usepackage{fancyvrb}
\newcommand{\VerbBar}{|}
\newcommand{\VERB}{\Verb[commandchars=\\\{\}]}
\DefineVerbatimEnvironment{Highlighting}{Verbatim}{commandchars=\\\{\}}
% Add ',fontsize=\small' for more characters per line
\usepackage{framed}
\definecolor{shadecolor}{RGB}{248,248,248}
\newenvironment{Shaded}{\begin{snugshade}}{\end{snugshade}}
\newcommand{\KeywordTok}[1]{\textcolor[rgb]{0.13,0.29,0.53}{\textbf{{#1}}}}
\newcommand{\DataTypeTok}[1]{\textcolor[rgb]{0.13,0.29,0.53}{{#1}}}
\newcommand{\DecValTok}[1]{\textcolor[rgb]{0.00,0.00,0.81}{{#1}}}
\newcommand{\BaseNTok}[1]{\textcolor[rgb]{0.00,0.00,0.81}{{#1}}}
\newcommand{\FloatTok}[1]{\textcolor[rgb]{0.00,0.00,0.81}{{#1}}}
\newcommand{\CharTok}[1]{\textcolor[rgb]{0.31,0.60,0.02}{{#1}}}
\newcommand{\StringTok}[1]{\textcolor[rgb]{0.31,0.60,0.02}{{#1}}}
\newcommand{\CommentTok}[1]{\textcolor[rgb]{0.56,0.35,0.01}{\textit{{#1}}}}
\newcommand{\OtherTok}[1]{\textcolor[rgb]{0.56,0.35,0.01}{{#1}}}
\newcommand{\AlertTok}[1]{\textcolor[rgb]{0.94,0.16,0.16}{{#1}}}
\newcommand{\FunctionTok}[1]{\textcolor[rgb]{0.00,0.00,0.00}{{#1}}}
\newcommand{\RegionMarkerTok}[1]{{#1}}
\newcommand{\ErrorTok}[1]{\textbf{{#1}}}
\newcommand{\NormalTok}[1]{{#1}}
\ifxetex
  \usepackage[setpagesize=false, % page size defined by xetex
              unicode=false, % unicode breaks when used with xetex
              xetex]{hyperref}
\else
  \usepackage[unicode=true]{hyperref}
\fi
\hypersetup{breaklinks=true,
            bookmarks=true,
            pdfauthor={David A Springate\^{}\{*1\}, Ivan Olier\^{}\{2\}, Evangelos Kontopantelis\^{}\{1\}; \^{}\{1\}Institute of Population Health, University of Manchester; \^{}\{2\}Institute of Biotechnology, University of Manchester; \^{}\{*\}Corresponding Author},
            pdftitle={Working with clinical code lists with rEHR},
            colorlinks=true,
            citecolor=blue,
            urlcolor=blue,
            linkcolor=magenta,
            pdfborder={0 0 0}}
\urlstyle{same}  % don't use monospace font for urls
\setlength{\parindent}{0pt}
\setlength{\parskip}{6pt plus 2pt minus 1pt}
\setlength{\emergencystretch}{3em}  % prevent overfull lines
\setcounter{secnumdepth}{0}

%%% Use protect on footnotes to avoid problems with footnotes in titles
\let\rmarkdownfootnote\footnote%
\def\footnote{\protect\rmarkdownfootnote}

%%% Change title format to be more compact
\usepackage{titling}
\setlength{\droptitle}{-2em}
  \title{Working with clinical code lists with rEHR}
  \pretitle{\vspace{\droptitle}\centering\huge}
  \posttitle{\par}
  \author{David A Springate\(^{*1}\), Ivan Olier\(^{2}\), Evangelos
Kontopantelis\(^{1}\) \\ \(^{1}\)Institute of Population Health, University of Manchester \\ \(^{2}\)Institute of Biotechnology, University of Manchester \\ \(^{*}\)Corresponding Author}
  \preauthor{\centering\large\emph}
  \postauthor{\par}
  \predate{\centering\large\emph}
  \postdate{\par}
  \date{2015-07-29}




\begin{document}

\maketitle

\begin{abstract}
The rEHR package provides functions to identify relevant code lists and
to automate the construction of clinical code lists.
\end{abstract}

\begin{Shaded}
\begin{Highlighting}[]
\KeywordTok{install.packages}\NormalTok{(}\StringTok{"devtools"}\NormalTok{)}
\KeywordTok{require}\NormalTok{(devtools)}
\KeywordTok{install_github}\NormalTok{(}\StringTok{"rEHR"}\NormalTok{, }\StringTok{"rpcdsearch"}\NormalTok{)}
\KeywordTok{require}\NormalTok{(rpcdsearch)}
\end{Highlighting}
\end{Shaded}

\section{Building draft definition
lists}\label{building-draft-definition-lists}

Definition lists can be defined for:

\begin{itemize}
\itemsep1pt\parskip0pt\parsep0pt
\item
  clinical terms (by either text search or searching for matching
  clinical codes)
\item
  test terms (by text search)
\item
  medications (by either text search or product code)
\end{itemize}

Building definition lists is a two stage process:

\begin{enumerate}
\def\labelenumi{\arabic{enumi}.}
\itemsep1pt\parskip0pt\parsep0pt
\item
  The search is defined by instantiating an object of class
  \texttt{MedicalDefinition}, containing the terms to be searched for in
  the lookup tables
\item
  A \texttt{definition\_search} is performed on the
  \texttt{MedicalDefinition} object and the relevant lookup tables to
  return a list of matching dataframes
\end{enumerate}

A \texttt{MedicalDefinition} object can be either made using terms
defined within \texttt{R} or with terms imported from an external csv
file

\subsection{Defining searches within
R}\label{defining-searches-within-r}

Use the \texttt{MedicalDefinition} constructor function to generate
search definitions. This takes the following arguments:

\begin{itemize}
\itemsep1pt\parskip0pt\parsep0pt
\item
  \texttt{terms} a list of character vectors representing clinical
  search terms or NULL
\item
  \texttt{codes} list of character vectors representing clinical code
  terms or NULL
\item
  \texttt{tests} list of character vectors representing test search
  terms or NULL
\item
  \texttt{drugs} list of character vectors representing drug search
  terms or NULL
\item
  \texttt{drugcodes} list of character vectors representing drug product
  code terms or NULL
\end{itemize}

vectors of length \textgreater{} 1 are searched for together (AND), in
any order. Different vectors in the same list are searched for
seperately (OR). Placing a ``-'' character at the start of a character
vector element excludes that terms from the search.

\begin{Shaded}
\begin{Highlighting}[]
\CommentTok{# vectors of length > 1 are combined as a single AND expression}
\CommentTok{# "-" excludes that term from the search}
\NormalTok{def <-}\StringTok{ }\KeywordTok{MedicalDefinition}\NormalTok{(}\DataTypeTok{terms =} \KeywordTok{list}\NormalTok{(}\StringTok{"peripheral vascular disease"}\NormalTok{, }\StringTok{"peripheral gangrene"}\NormalTok{, }\StringTok{"-wrong answer"}\NormalTok{,}
                      \StringTok{"intermittent claudication"}\NormalTok{, }\StringTok{"thromboangiitis obliterans"}\NormalTok{,}
                      \StringTok{"thromboangiitis obliterans"}\NormalTok{, }\StringTok{"diabetic peripheral angiopathy"}\NormalTok{,}
                      \KeywordTok{c}\NormalTok{(}\StringTok{"diabetes"}\NormalTok{, }\StringTok{"peripheral angiopathy"}\NormalTok{),  }\CommentTok{# combined as a single AND expression}
                      \KeywordTok{c}\NormalTok{(}\StringTok{"diabetes"}\NormalTok{, }\StringTok{"peripheral angiopathy"}\NormalTok{),}
                      \KeywordTok{c}\NormalTok{(}\StringTok{"buerger"}\NormalTok{,  }\StringTok{"disease presenile_gangrene"}\NormalTok{),}
                      \StringTok{"thromboangiitis obliterans"}\NormalTok{,}
                      \StringTok{"-rubbish"}\NormalTok{, }\CommentTok{# exclusion}
                      \KeywordTok{c}\NormalTok{(}\StringTok{"percutaneous_transluminal_angioplasty"}\NormalTok{, }\StringTok{"artery"}\NormalTok{),}
                      \KeywordTok{c}\NormalTok{(}\StringTok{"bypass"}\NormalTok{, }\StringTok{"iliac_artery"}\NormalTok{),}
                      \KeywordTok{c}\NormalTok{(}\StringTok{"bypass"}\NormalTok{, }\StringTok{"femoral_artery"}\NormalTok{),}
                      \KeywordTok{c}\NormalTok{(}\StringTok{"femoral_artery"} \NormalTok{, }\StringTok{"occlusion"}\NormalTok{),}
                      \KeywordTok{c}\NormalTok{(}\StringTok{"popliteal_artery"}\NormalTok{, }\StringTok{"occlusion"}\NormalTok{),}
                      \StringTok{"dissecting_aortic_aneurysm"}\NormalTok{, }\StringTok{"peripheral_angiopathic_disease"}\NormalTok{,}
                      \StringTok{"acrocyanosis"}\NormalTok{, }\StringTok{"acroparaesthesia"}\NormalTok{, }\StringTok{"erythrocyanosis"}\NormalTok{,}
                      \StringTok{"erythromelalgia"}\NormalTok{, }\StringTok{"ABPI"}\NormalTok{,}
                      \KeywordTok{c}\NormalTok{(}\StringTok{"ankle"}\NormalTok{, }\StringTok{"brachial"}\NormalTok{),}
                      \KeywordTok{c}\NormalTok{(}\StringTok{"ankle"}\NormalTok{, }\StringTok{"pressure"}\NormalTok{),}
                      \KeywordTok{c}\NormalTok{(}\StringTok{"left"}\NormalTok{, }\StringTok{"brachial"}\NormalTok{),}
                      \KeywordTok{c}\NormalTok{(}\StringTok{"left"}\NormalTok{, }\StringTok{"pressure"}\NormalTok{),}
                      \KeywordTok{c}\NormalTok{(}\StringTok{"right"}\NormalTok{, }\StringTok{"brachial"}\NormalTok{),}
                      \KeywordTok{c}\NormalTok{(}\StringTok{"right"}\NormalTok{, }\StringTok{"pressure"}\NormalTok{)),}
         \DataTypeTok{codes =} \KeywordTok{list}\NormalTok{(}\StringTok{"G73"}\NormalTok{),}
         \DataTypeTok{tests =} \OtherTok{NULL}\NormalTok{,}
         \DataTypeTok{drugs =} \KeywordTok{list}\NormalTok{(}\StringTok{"insulin"}\NormalTok{, }\StringTok{"diabet"}\NormalTok{, }\StringTok{"aspirin"}\NormalTok{))}
\end{Highlighting}
\end{Shaded}

When searching for codes, a range of clinical codes can be searched for
by providing two codes seperated by a hyphen. e.g. \texttt{E114-E117z}.

\subsection{importing searches via a csv
file}\label{importing-searches-via-a-csv-file}

Searches can be imported from a csv file in
\href{https://github.com/rOpenHealth/rpcdsearch/blob/master/inst/extdata/example_search.csv}{this
format}

The first column in every row determines the list that the term applies
to and the second column determines whether the term should be included
or excluded. Note that the csv does not have to be a valid format for
conversion to a dataframe. Extra columns can be used to include terms to
be combined as an AND expression with the other terms on that row. The
title row can also be ommitted. You can use standard regex escape
patterns in the term definitions.

The data is called into \texttt{R} in the following way:

\begin{Shaded}
\begin{Highlighting}[]
\NormalTok{## Using the example search definition provided with the package}
\NormalTok{def2 <-}\StringTok{ }\KeywordTok{import_definitions}\NormalTok{(}\KeywordTok{system.file}\NormalTok{(}\StringTok{"extdata"}\NormalTok{, }\StringTok{"example_search.csv"}\NormalTok{, }\DataTypeTok{package =} \StringTok{"rpcdsearch"}\NormalTok{))}
\end{Highlighting}
\end{Shaded}

\subsection{Running searches}\label{running-searches}

Once a search has been defined, the relevant lookup tables should be
called in. Note that these lookup tables are not provided with the
package and will be specific to the users EHR database. These examples
are using CPRD lookups and EHR definitions (See the
\href{https://github.com/rOpenHealth/rpcdsearch/blob/master/R/ehr_system.R}{ehr\_system}
code for details of how the interface with CPRD is implemented).

\begin{Shaded}
\begin{Highlighting}[]
\NormalTok{## Use fileEncoding="latin1" to avoid any issues with non-ascii characters}
\NormalTok{medical_table <-}\StringTok{ }\KeywordTok{read.delim}\NormalTok{(}\StringTok{"Lookups//medical.txt"}\NormalTok{, }\DataTypeTok{fileEncoding=}\StringTok{"latin1"}\NormalTok{, }\DataTypeTok{stringsAsFactors =} \OtherTok{FALSE}\NormalTok{)}
\NormalTok{drug_table <-}\StringTok{ }\KeywordTok{read.delim}\NormalTok{(}\StringTok{"Lookups/product.txt"}\NormalTok{, }\DataTypeTok{fileEncoding=}\StringTok{"latin1"}\NormalTok{, }\DataTypeTok{stringsAsFactors =} \OtherTok{FALSE}\NormalTok{)}
\end{Highlighting}
\end{Shaded}

And the search can be run:

\begin{Shaded}
\begin{Highlighting}[]
\NormalTok{draft_lists <-}\StringTok{ }\KeywordTok{build_definition_lists}\NormalTok{(def, }\DataTypeTok{medical_table =} \NormalTok{medical_table,}\DataTypeTok{drug_table =} \NormalTok{drug_table)}
\end{Highlighting}
\end{Shaded}

This returns a list of dataframes for each of the provided search lists.
If \texttt{terms} and \texttt{codes} are provided in the definition, it
also contains a \texttt{combined\_terms\_codes} data frame which is a
combination of \texttt{terms} and \texttt{codes} with duplicate rows
removed.

\section{Exporting code lists}\label{exporting-code-lists}

The code lists produced by \texttt{build\_definition\_lists} will often
want to be reviewed by clinicians or non-technical researchers. To
facilitate this, there is an \texttt{export\_definition\_search}
function to export the code lists as an Excel file, with each list
occupying a tab in the file. To export a code list:

\begin{Shaded}
\begin{Highlighting}[]
\NormalTok{out_file <-}\StringTok{ "def_searches.xlsx"}
\KeywordTok{export_definition_search}\NormalTok{(draft_lists, out_file)}
\end{Highlighting}
\end{Shaded}

\end{document}
